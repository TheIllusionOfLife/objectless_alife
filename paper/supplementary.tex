\documentclass[letterpaper]{article}
\usepackage{natbib,alifeconf}  %% The order is important
\usepackage{url,hyperref}
\usepackage{amsmath,amssymb}
\usepackage{booktabs}
\usepackage{xcolor}
\usepackage{subcaption}
\usepackage{multirow}

\title{Supplementary Material:\\
Emergent Spatial Coordination from Negative Selection Alone}

\author{Anonymous}

\begin{document}

\maketitle

%% =========================================================================
%% A. DENSITY SWEEP ROBUSTNESS
%% =========================================================================
\section{Density Sweep Robustness}
\label{app:density_sweep}

The main experiments use a fixed density of 7.5\% (30 agents on a
$20 \times 20$ grid). To assess robustness across density levels, we
evaluated both Phase~1 and Phase~2 across 12 density conditions: 3 grid
sizes ($15 \times 15$, $20 \times 20$, $30 \times 30$) $\times$ 4 agent
counts (15, 30, 60, 90), yielding densities from 0.017 to 0.400.
Each condition was evaluated with 600 rules (100 rules $\times$ 6 seed
batches), totaling 14{,}400 rule evaluations, all using the Miller-Madow
bias-corrected MI estimator.

\begin{table*}[htbp]
\centering
\caption{Density sweep results across 12 conditions.
$\mathrm{MI}_{\mathrm{excess}}$ values are
Miller-Madow corrected, bits.  Phase~2 achieves nonzero median
$\mathrm{MI}_{\mathrm{excess}}$ in 8 of 12 conditions, including all
conditions with $\geq 60$ agents, while Phase~1 remains at zero across
all densities tested.}
\label{tab:density_sweep}
\begin{tabular}{rllcccc}
\toprule
Density & Grid & Agents & \multicolumn{2}{c}{Median MI\textsubscript{excess} (bits)} & \multicolumn{2}{c}{Survival (\%)} \\
\cmidrule(lr){4-5} \cmidrule(lr){6-7}
 & & & P1 & P2 & P1 & P2 \\
\midrule
0.017 & $30 \times 30$ & 15 & 0.000 & 0.000 & 86.7 & 86.7 \\
0.033 & $30 \times 30$ & 30 & 0.000 & 0.000 & 83.3 & 85.8 \\
0.037 & $20 \times 20$ & 15 & 0.000 & 0.000 & 80.8 & 81.8 \\
0.067 & $15 \times 15$ & 15 & 0.000 & 0.000 & 68.7 & 76.0 \\
0.067 & $30 \times 30$ & 60 & 0.000 & 0.199 & 70.5 & 76.3 \\
0.075 & $20 \times 20$ & 30 & 0.000 & 0.114 & 70.2 & 72.5 \\
0.100 & $30 \times 30$ & 90 & 0.000 & 0.246 & 67.0 & 74.2 \\
0.133 & $15 \times 15$ & 30 & 0.000 & 0.130 & 62.2 & 64.8 \\
0.150 & $20 \times 20$ & 60 & 0.000 & 0.191 & 63.0 & 69.7 \\
0.225 & $20 \times 20$ & 90 & 0.002 & 0.176 & 64.0 & 76.7 \\
0.267 & $15 \times 15$ & 60 & 0.000 & 0.148 & 63.5 & 72.7 \\
0.400 & $15 \times 15$ & 90 & 0.000 & 0.107 & 68.0 & 83.7 \\
\bottomrule
\end{tabular}
\end{table*}

Table~\ref{tab:density_sweep} and Figures~\ref{fig:mi_density}
and~\ref{fig:surv_density} present the results. Phase~2 achieves nonzero
median $\mathrm{MI}_{\mathrm{excess}}$ in 8 of 12 conditions, including all
conditions with $\geq 60$ agents, peaking near $d = 0.100$
($\mathrm{MI}_{\mathrm{excess}} = 0.246$ bits at $30 \times 30$, 90 agents)
before declining at higher densities. Phase~1 remains at zero median
$\mathrm{MI}_{\mathrm{excess}}$ across all 12 conditions, confirming that the
Phase~2 advantage is not an artifact of the specific grid configuration used
in the main experiments.

Phase~2 also consistently achieves higher survival rates than Phase~1, with
the gap widening at higher densities (e.g., 76.7\% vs.\ 64.0\% at $d =
0.225$; 83.7\% vs.\ 68.0\% at $d = 0.400$).

\begin{figure}[htbp]
\centering
\includegraphics[width=0.85\linewidth]{figures/figA1_mi_vs_density.pdf}
\caption{Median neighbor MI vs.\ agent density for Phase~1 and Phase~2.
Phase~2 MI peaks at medium densities and declines at higher densities, while
Phase~1 remains at zero throughout.}
\label{fig:mi_density}
\end{figure}

\begin{figure}[htbp]
\centering
\includegraphics[width=0.85\linewidth]{figures/figA2_survival_vs_density.pdf}
\caption{Survival rate vs.\ agent density. Phase~2 consistently achieves
higher survival than Phase~1, with the gap widening at higher densities.}
\label{fig:surv_density}
\end{figure}

%% =========================================================================
%% B. MULTI-SEED ROBUSTNESS
%% =========================================================================
\section{Multi-Seed Robustness}
\label{app:multi_seed}

The main experiments evaluate each rule table with a single simulation seed.
To assess whether $\mathrm{MI}_{\mathrm{excess}}$ levels are robust
properties of the rule table rather than seed-specific accidents, we selected
the top~50 rules by $\mathrm{MI}_{\mathrm{excess}}$ from each rule-based
condition (Phase~2, Phase~1, and Control) and re-evaluated each across
20 independent simulation seeds.

\begin{table*}[htbp]
\centering
\caption{Multi-seed robustness of top-50 rules per condition (20 seeds each).
Phase~2's elevated MI is a robust rule property; Phase~1 and Control's
low/zero MI are equally stable across seeds.}
\label{tab:multi_seed}
\begin{tabular}{lccc}
\toprule
Metric & Phase~2 & Phase~1 & Control \\
\midrule
Rules with median $\mathrm{MI}_{\mathrm{excess}} > 0$
  & 41/50 (82\%) & 18/50 (36\%) & 22/50 (44\%) \\
Mean $P(\mathrm{MI}_{\mathrm{excess}} > 0)$ across seeds
  & 0.733 & 0.396 & 0.446 \\
Overall survival rate
  & 76.2\% & 90.7\% & 78.7\% \\
\bottomrule
\end{tabular}
\end{table*}

Table~\ref{tab:multi_seed} shows that 82\% of top Phase~2 rules maintain
positive median $\mathrm{MI}_{\mathrm{excess}}$ across seeds, and on average
73.3\% of seeds per rule produce positive excess MI.  In contrast, only 36\%
of top Phase~1 rules and 44\% of top Control rules maintain positive median
$\mathrm{MI}_{\mathrm{excess}}$, with mean positive fractions of 0.396 and
0.446 respectively.  This confirms that MI differences across conditions
reflect genuine rule properties rather than seed-specific accidents: Phase~2's
elevated MI is robust, while Phase~1 and Control's lower MI levels are equally
stable across seeds.

%% =========================================================================
%% C. MORAN'S I (RELOCATED FROM TABLE 1)
%% =========================================================================
\section{Moran's $I$ by Condition}
\label{app:morans_i}

Table~1 in the main text reports the same-state adjacency fraction as the
primary categorical spatial statistic.  For completeness, we report
Moran's~$I$ here, noting that it treats categorical states as numeric
(computing deviations from an arithmetic mean) and is therefore
inappropriate as a primary indicator for nominal data.

\begin{table}[htbp]
\centering
\caption{Median Moran's $I$ by condition (final-step snapshot,
5{,}000 rules per condition).  Moran's $I$ treats states as numeric
and is a secondary indicator; see Table~1 for the categorical
adjacency fraction.}
\label{tab:morans_i}
\begin{tabular}{lc}
\toprule
Condition & Median Moran's $I$ \\
\midrule
Random Walk & $-0.030$ \\
Control     & 0.124 \\
Phase 1     & $-0.011$ \\
Phase 2     & $-0.020$ \\
\bottomrule
\end{tabular}
\end{table}

%% =========================================================================
%% D. HALT WINDOW SENSITIVITY
%% =========================================================================
\section{Halt Window Sensitivity}
\label{app:halt_window}

The main experiments use a 10-step halt window.  To assess sensitivity,
we evaluated the top-50 Phase~2 rules across halt windows of
\{5, 10, 20\} steps.

\begin{table}[htbp]
\centering
\caption{Halt window sensitivity for top-50 Phase~2 rules.
Results are qualitatively unchanged across the tested range.}
\label{tab:halt_window}
\begin{tabular}{ccc}
\toprule
Halt Window & Survival Rate & Median MI\textsubscript{excess} \\
\midrule
5  & 78.0\% & 0.486 \\
10 & 78.0\% & 0.486 \\
20 & 78.0\% & 0.486 \\
\bottomrule
\end{tabular}
\end{table}

Table~\ref{tab:halt_window} confirms that the halt-window parameter has
no impact on the qualitative findings for these top-performing rules:
survival rates (78.0\%) and median MI\textsubscript{excess} (0.486~bits)
are identical across all three tested windows.  This indicates that the
top-50 rules either survive to completion or halt well within the first
5~steps, with no rules in the intermediate regime.

%% =========================================================================
%% E. SURVIVAL RATES WITH CIs
%% =========================================================================
\section{Survival Rates with Confidence Intervals}
\label{app:survival_ci}

\begin{table}[htbp]
\centering
\caption{Survival rates with Wilson score 95\% confidence intervals
(5{,}000 rules per condition).}
\label{tab:survival_ci}
\begin{tabular}{lccc}
\toprule
Condition & Survived / Total & Rate & 95\% CI \\
\midrule
Random Walk & 5000/5000 & 100.0\% & [99.9, 100.0]\% \\
Control     & 2226/5000 & 44.5\%  & [43.1, 45.9]\% \\
Phase 1     & 3571/5000 & 71.4\%  & [70.1, 72.7]\% \\
Phase 2     & 3735/5000 & 74.7\%  & [73.5, 75.9]\% \\
\bottomrule
\end{tabular}
\end{table}

%% =========================================================================
%% F. RANDOM WALK DENSITY SWEEP
%% =========================================================================
\section{Random Walk MI\textsubscript{excess} Across Densities}
\label{app:rw_density}

To confirm that the random walk's $\mathrm{MI}_{\mathrm{excess}}$ remains
near zero regardless of agent density, we extended the density sweep to
include the Random Walk condition.  Across all 12 density conditions
(density range 0.017--0.400), the random walk produces
$\mathrm{MI}_{\mathrm{excess}} \approx 0$ (median $\leq 0.06$~bits),
confirming that its elevated raw MI is entirely attributable to
pair-count bias at all tested densities.

%% =========================================================================
%% G. ALTERNATIVE NULL MODELS
%% =========================================================================
\section{Alternative Null Models}
\label{app:alt_nulls}

In addition to the state-shuffle null used throughout the main text, we
evaluated two alternative null models to assess the robustness of the MI
calibration:

\begin{itemize}
  \item \textbf{Block shuffle}: States are shuffled within spatial blocks
        ($4 \times 4$), preserving local autocorrelation structure while
        destroying inter-block correlations.
  \item \textbf{Fixed-marginal}: Synthetic snapshots are generated with
        identical marginal state distributions but independent spatial
        placement (each position draws independently from the observed
        state frequencies).
\end{itemize}

\begin{table}[htbp]
\centering
\caption{Alternative null model comparison for top-50 Phase~2 rules
(mean MI across 200 null samples per rule).  All three null models
produce substantially lower MI than the observed values, confirming
that Phase~2's elevated MI reflects genuine spatial coordination.}
\label{tab:alt_nulls}
\begin{tabular}{lc}
\toprule
Null Model & Mean Null MI (bits) \\
\midrule
State shuffle (main text) & 0.264 \\
Block shuffle ($4 \times 4$) & 0.899 \\
Fixed-marginal & 0.250 \\
\bottomrule
\end{tabular}
\end{table}

The block-shuffle null produces substantially higher MI (0.899~bits) than
the state-shuffle null (0.264~bits), as expected since it preserves
within-block correlations.  The fixed-marginal null (0.250~bits) is
comparable to the state-shuffle.  In all cases, mean observed MI for the
top-50 Phase~2 rules (1.646~bits) substantially exceeds the null values,
confirming genuine spatial coordination.

%% =========================================================================
%% H. SPATIAL SCRAMBLING
%% =========================================================================
\section{Spatial Scrambling Control}
\label{app:spatial_scrambling}

To confirm that Phase~2's elevated MI depends on agents' specific
positions rather than their state distribution alone, we performed
spatial scrambling: for each rule's final snapshot, we randomly
reassigned occupied positions among agents while keeping their states
fixed ($N = 200$ scrambles per rule).

For top-50 Phase~2 rules, the mean observed MI is 1.646~bits while the
mean scrambled MI drops to 0.270~bits---comparable to the shuffle null
baseline (0.264~bits).  This confirms that the observed MI arises from
genuine local spatial coordination (agents with correlated states being
\emph{near} each other) rather than from the state distribution itself.

%% =========================================================================
%% I. TRANSFER ENTROPY
%% =========================================================================
\section{Transfer Entropy}
\label{app:transfer_entropy}

Mutual information measures symmetric statistical dependence between
neighboring states.  To assess \emph{directional} information flow, we
computed transfer entropy (TE) from neighbor states to agent next-states:
\begin{equation}
  \mathrm{TE} = I(S_j^t ; S_i^{t+1} \mid S_i^t)
\end{equation}
where $S_i^t$ is agent $i$'s state at time $t$ and $S_j^t$ is a
neighboring agent's state.  This measures how much knowing a neighbor's
current state reduces uncertainty about the focal agent's next state,
beyond what the agent's own current state provides.

Miller-Madow bias correction is applied.  For the top-50 rules in each
condition, median TE values are: Phase~2 = 0.072~bits, Phase~1 =
0.003~bits, Control = 0.113~bits.  Phase~2 shows substantially elevated
TE compared to Phase~1, confirming directional information flow from
neighbors to agents.  The Control condition's higher TE reflects its
inclusion of a step-clock dimension that creates temporal state
dependence without genuine spatial coordination (recall that Control
$\mathrm{MI}_{\mathrm{excess}} \approx 0$).

%% =========================================================================
%% J. CAPACITY-MATCHED CONTROLS
%% =========================================================================
\section{Capacity-Matched Controls}
\label{app:capacity_matched}

To further isolate the role of observation \emph{content} from table
\emph{capacity}, we evaluated two additional control conditions:

\begin{itemize}
  \item \textbf{Capacity-matched Phase~1}: 100-entry tables where
        indices are aliased so that all dominant-state values for the
        same (self-state, neighbor-count) pair map to the same action.
        This provides Phase-2-sized tables with only Phase-1-level
        observation content.
  \item \textbf{Random-encoding Phase~2}: 100-entry tables with the
        same alphabet size as Phase~2, but the mapping from neighborhood
        configuration to observation index is randomly permuted.  This
        tests whether the \emph{structure} of the encoding matters
        beyond alphabet size.
\end{itemize}

The capacity-matched Phase~1 control produces median
$\mathrm{MI}_{\mathrm{excess}} = 0.000$ (survival 71.4\%), matching
standard Phase~1 ($\mathrm{MI}_{\mathrm{excess}} = 0.000$) and well below
standard Phase~2 ($\mathrm{MI}_{\mathrm{excess}} = 0.096$).  This
confirms that table capacity alone does not explain the Phase~2
advantage---Phase-1-level observations remain ineffective even with
100-entry tables.  The random-encoding Phase~2 control produces median
$\mathrm{MI}_{\mathrm{excess}} = 0.106$ (survival 75.1\%), comparable to
standard Phase~2, which is expected because the observation encoding
(self-state, neighbor count, dominant state) is identical; only the
table-entry ordering differs, which is irrelevant for randomly generated
tables.

%% =========================================================================
%% K. PSEUDOCODE
%% =========================================================================
\section{Algorithmic Pseudocode}
\label{app:pseudocode}

\paragraph{Simulation loop} (\texttt{world.py:step()}).
At each of the 200 time steps:
\begin{enumerate}
  \item Generate a random permutation of agent indices.
  \item For each agent in order:
    \begin{enumerate}
      \item Observe local neighborhood (von~Neumann, 4 cells).
      \item Compute observation vector $(s, n, d)$ or $(s, n)$ depending on
            phase, where $s$ = own state, $n$ = occupied neighbor count,
            $d$ = dominant neighbor state.
      \item Look up action in shared rule table: $a = T[\mathrm{index}(s, n, d)]$.
      \item Execute action: move (if target cell empty), change state, or no-op.
    \end{enumerate}
\end{enumerate}

\paragraph{Filter checks} (\texttt{filters.py}).
After each step, check:
(1) \emph{Halt}: positions and states unchanged for 10 consecutive steps $\to$ terminate.
(2) \emph{State uniform}: all agents share the same state $\to$ terminate.

\paragraph{MI computation} (\texttt{metrics.py}).
For the final snapshot:
\begin{enumerate}
  \item Enumerate all occupied neighbor pairs (right/down on torus, deduped).
  \item Compute joint and marginal state distributions from pairs.
  \item $\hat{I} = \sum p(s_i, s_j) \log_2 \frac{p(s_i, s_j)}{p(s_i)\,p(s_j)}$.
  \item Apply Miller-Madow correction:
        $\hat{I}_{\mathrm{MM}} = \hat{I} - \frac{K_{\mathrm{joint}} - K_X - K_Y + 1}{2n\ln 2}$.
  \item Clamp to $\geq 0$.
\end{enumerate}

\paragraph{Shuffle null} (\texttt{metrics.py}).
Repeat 200 times: permute states among fixed occupied positions, compute
$\hat{I}_{\mathrm{MM}}$, average.
$\mathrm{MI}_{\mathrm{excess}} = \max(\hat{I}_{\mathrm{MM}} - \overline{I}_{\mathrm{shuffle}}, 0)$.

%% =========================================================================
%% L. CROSS-CONDITION METRIC PROFILES
%% =========================================================================
\section{Cross-Condition Metric Profiles}
\label{app:metric_profiles}

Beyond mutual information, the simulation records five additional metric
families (seven individual metrics) at every time step.
Table~\ref{tab:metric_profiles} reports final-step values for surviving
rules across all four conditions.  All pairwise comparisons
(Mann-Whitney $U$, Holm-Bonferroni corrected) are significant at
$p < 0.001$ for every metric.

\begin{table*}[htbp]
\centering
\caption{Cross-condition metric profiles (surviving rules, final-step
values).  Median [Q1, Q3] reported.  All pairwise Mann-Whitney $U$
tests (Holm-Bonferroni corrected) are significant at $p < 0.001$;
effect sizes are reported in the text as Cliff's $|\delta|$.}
\label{tab:metric_profiles}
\begin{tabular}{lcccc}
\toprule
Metric & Random Walk & Control & Phase~1 & Phase~2 \\
\midrule
Compression ratio
  & 0.178 [0.173, 0.180] & 0.163 [0.158, 0.168] & 0.160 [0.153, 0.168] & 0.160 [0.153, 0.168] \\
Action entropy (mean)
  & 3.141 [3.139, 3.143] & 2.725 [2.555, 2.853] & 0.621 [0.337, 0.958] & 0.778 [0.445, 1.249] \\
Action entropy (var.)
  & 0.000 [0.000, 0.000] & 0.020 [0.008, 0.041] & 0.152 [0.071, 0.234] & 0.190 [0.112, 0.278] \\
Cluster count
  & 29 [28, 30] & 29 [28, 30] & 26 [22, 29] & 27 [24, 29] \\
Quasi-period.\ peaks
  & 4 [3, 5] & 15 [9, 19] & 0 [0, 6] & 2 [0, 7] \\
Phase trans.\ $\max \Delta$
  & 0.266 [0.233, 0.309] & 0.691 [0.572, 0.818] & 0.211 [0.161, 0.327] & 0.260 [0.173, 0.361] \\
State entropy
  & 1.941 [1.897, 1.969] & 1.157 [0.948, 1.446] & 1.295 [0.922, 1.555] & 1.438 [1.091, 1.693] \\
\bottomrule
\end{tabular}
\end{table*}

\paragraph{Role differentiation.}
The variance of per-agent action entropy (action\_entropy\_variance)
captures the degree to which agents specialize into distinct behavioral
roles.  High variance indicates that some agents repeatedly select the
same action while others explore diverse actions---a signature of
emergent role differentiation.  Phase~2 exhibits the highest action
entropy variance (median 0.190), followed by Phase~1 (0.152), while
Control (0.020) and Random Walk ($< 0.001$) show minimal differentiation.
The Phase~1 vs.\ Phase~2 difference is significant (Cliff's
$|\delta| = 0.185$, $p < 10^{-42}$), confirming that richer
observations support greater role specialization.

\paragraph{Temporal signatures.}
Quasi-periodicity peak count and phase-transition $\max \Delta$ capture
temporal dynamics beyond the MI time-series snapshots in the main text.
Control shows strikingly high quasi-periodicity (median 15 peaks) and
phase-transition $\max \Delta$ (median 0.691), far exceeding Phase~1
(0 peaks, 0.211) and Phase~2 (2 peaks, 0.260).  This reflects
Control's step-clock dimension, which drives periodic state cycling
without genuine spatial coordination ($\mathrm{MI}_{\mathrm{excess}}
\approx 0$).  Phase~1 and Phase~2 show low quasi-periodicity,
consistent with their spatially structured but temporally stable
dynamics.  The Phase~1 vs.\ Phase~2 difference in $\max \Delta$ is
significant (Cliff's $|\delta| = 0.119$, $p < 10^{-18}$), suggesting
that Phase~2's richer observations produce slightly more dynamic
temporal trajectories.

%% =========================================================================
%% M. CASCADED FILTER ANALYSIS
%% =========================================================================
\section{Cascaded Filter Analysis}
\label{app:cascaded_filters}

The main experiments use only weak (viability) filters: halt detection
and state uniformity.  The codebase also implements medium-strength
filters---short-period detection (period $\leq 2$, checked over 8
snapshots) and low-activity detection (unique-action ratio $< 0.2$ over
5 steps)---which target dynamically trivial but non-halted simulations.

To assess how filter stringency affects the survivor pool, we re-ran
all 5{,}000 rules per condition with both weak and medium filters
enabled (same deterministic seeds, ensuring direct comparability).
Table~\ref{tab:cascade_survival} reports the cascade survival counts.

\begin{table}[htbp]
\centering
\caption{Cascade survival table: weak-only vs.\ weak+medium filters
(5{,}000 rules per condition, same seeds).  Medium filters further
refine the survivor pool while the observation-richness ordering
(Control $<$ P1 $<$ P2) persists.}
\label{tab:cascade_survival}
\begin{tabular}{lccc}
\toprule
Condition & Weak Only & Weak+Medium & $\Delta$ \\
\midrule
Phase~1 & 3{,}571 (71.4\%) & 2{,}812 (56.2\%) & $-759$ \\
Phase~2 & 3{,}735 (74.7\%) & 2{,}906 (58.1\%) & $-829$ \\
Control & 2{,}226 (44.5\%) & 1{,}915 (38.3\%) & $-311$ \\
\bottomrule
\end{tabular}
\end{table}

Medium filters remove an additional 15--17\% of rules in Phase~1 and
Phase~2, and 6\% in Control (which already has lower weak-filter
survival).  Crucially, the observation-richness ordering persists:
Phase~2 retains the highest survival rate (58.1\%) among medium-filter
survivors, followed by Phase~1 (56.2\%) and Control (38.3\%).
Furthermore, the median $\mathrm{MI}_{\mathrm{excess}}$ among Phase~2
medium-filter survivors is 0.153~bits, confirming that the MI advantage
is not an artifact of lax filtering.  Phase~1 and Control
medium-filter survivors have median $\mathrm{MI}_{\mathrm{excess}} = 0$.

%% =========================================================================
%% BIBLIOGRAPHY
%% =========================================================================
\footnotesize
\bibliographystyle{apalike}
\bibliography{references}

\end{document}
