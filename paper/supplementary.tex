\documentclass[letterpaper]{article}
\usepackage{natbib,alifeconf}  %% The order is important
\usepackage{url,hyperref}
\usepackage{amsmath,amssymb}
\usepackage{booktabs}
\usepackage{xcolor}
\usepackage{subcaption}
\usepackage{multirow}

\title{Supplementary Material:\\
Emergent Spatial Coordination from Negative Selection Alone}

\author{Anonymous}

\begin{document}

\maketitle

%% =========================================================================
%% A. DENSITY SWEEP ROBUSTNESS
%% =========================================================================
\section{Density Sweep Robustness}
\label{app:density_sweep}

The main experiments use a fixed density of 7.5\% (30 agents on a
$20 \times 20$ grid). To assess robustness across density levels, we
evaluated both Phase~1 and Phase~2 across 12 density conditions: 3 grid
sizes ($15 \times 15$, $20 \times 20$, $30 \times 30$) $\times$ 4 agent
counts (15, 30, 60, 90), yielding densities from 0.017 to 0.400.
Each condition was evaluated with 600 rules (100 rules $\times$ 6 seed
batches), totaling 14{,}400 rule evaluations, all using the Miller-Madow
bias-corrected MI estimator.

\begin{table*}[htbp]
\centering
\caption{Density sweep results across 12 conditions.
$\mathrm{MI}_{\mathrm{excess}}$ values are
Miller-Madow corrected, bits.  Phase~2 achieves nonzero median
$\mathrm{MI}_{\mathrm{excess}}$ in 8 of 12 conditions, including all
conditions with $\geq 60$ agents, while Phase~1 remains at zero across
all densities tested.}
\label{tab:density_sweep}
\begin{tabular}{rllcccc}
\toprule
Density & Grid & Agents & \multicolumn{2}{c}{Median MI\textsubscript{excess} (bits)} & \multicolumn{2}{c}{Survival (\%)} \\
\cmidrule(lr){4-5} \cmidrule(lr){6-7}
 & & & P1 & P2 & P1 & P2 \\
\midrule
0.017 & $30 \times 30$ & 15 & 0.000 & 0.000 & 86.7 & 86.7 \\
0.033 & $30 \times 30$ & 30 & 0.000 & 0.000 & 83.3 & 85.8 \\
0.037 & $20 \times 20$ & 15 & 0.000 & 0.000 & 80.8 & 81.8 \\
0.067 & $15 \times 15$ & 15 & 0.000 & 0.000 & 68.7 & 76.0 \\
0.067 & $30 \times 30$ & 60 & 0.000 & 0.199 & 70.5 & 76.3 \\
0.075 & $20 \times 20$ & 30 & 0.000 & 0.114 & 70.2 & 72.5 \\
0.100 & $30 \times 30$ & 90 & 0.000 & 0.246 & 67.0 & 74.2 \\
0.133 & $15 \times 15$ & 30 & 0.000 & 0.130 & 62.2 & 64.8 \\
0.150 & $20 \times 20$ & 60 & 0.000 & 0.191 & 63.0 & 69.7 \\
0.225 & $20 \times 20$ & 90 & 0.002 & 0.176 & 64.0 & 76.7 \\
0.267 & $15 \times 15$ & 60 & 0.000 & 0.148 & 63.5 & 72.7 \\
0.400 & $15 \times 15$ & 90 & 0.000 & 0.107 & 68.0 & 83.7 \\
\bottomrule
\end{tabular}
\end{table*}

Table~\ref{tab:density_sweep} and Figures~\ref{fig:mi_density}
and~\ref{fig:surv_density} present the results. Phase~2 achieves nonzero
median $\mathrm{MI}_{\mathrm{excess}}$ in 8 of 12 conditions, including all
conditions with $\geq 60$ agents, peaking near $d = 0.100$
($\mathrm{MI}_{\mathrm{excess}} = 0.246$ bits at $30 \times 30$, 90 agents)
before declining at higher densities. Phase~1 remains at zero median
$\mathrm{MI}_{\mathrm{excess}}$ across all 12 conditions, confirming that the
Phase~2 advantage is not an artifact of the specific grid configuration used
in the main experiments.

Phase~2 also consistently achieves higher survival rates than Phase~1, with
the gap widening at higher densities (e.g., 76.7\% vs.\ 64.0\% at $d =
0.225$; 83.7\% vs.\ 68.0\% at $d = 0.400$).

\begin{figure}[htbp]
\centering
\includegraphics[width=0.85\linewidth]{figures/figA1_mi_vs_density.pdf}
\caption{Median neighbor MI vs.\ agent density for Phase~1 and Phase~2.
Phase~2 MI peaks at medium densities and declines at higher densities, while
Phase~1 remains at zero throughout.}
\label{fig:mi_density}
\end{figure}

\begin{figure}[htbp]
\centering
\includegraphics[width=0.85\linewidth]{figures/figA2_survival_vs_density.pdf}
\caption{Survival rate vs.\ agent density. Phase~2 consistently achieves
higher survival than Phase~1, with the gap widening at higher densities.}
\label{fig:surv_density}
\end{figure}

%% =========================================================================
%% B. MULTI-SEED ROBUSTNESS
%% =========================================================================
\section{Multi-Seed Robustness}
\label{app:multi_seed}

The main experiments evaluate each rule table with a single simulation seed.
To assess whether high-$\mathrm{MI}_{\mathrm{excess}}$ rules are robust
properties of the rule table rather than seed-specific accidents, we selected
the top~50 Phase~2 rules by $\mathrm{MI}_{\mathrm{excess}}$ from the main
experiment and re-evaluated each across 20 independent simulation seeds.

\begin{table}[htbp]
\centering
\caption{Multi-seed robustness of top-50 Phase~2 rules (20 seeds each).}
\label{tab:multi_seed}
\begin{tabular}{lc}
\toprule
Metric & Value \\
\midrule
Rules with median $\mathrm{MI}_{\mathrm{excess}} > 0$ & 41/50 (82.0\%) \\
Mean $P(\mathrm{MI}_{\mathrm{excess}} > 0)$ across seeds & 0.733 \\
Overall survival rate & 76.2\% \\
\bottomrule
\end{tabular}
\end{table}

Table~\ref{tab:multi_seed} shows that 82\% of top Phase~2 rules maintain
positive median $\mathrm{MI}_{\mathrm{excess}}$ across seeds, and on average
73.3\% of seeds per rule produce positive excess MI. This confirms that
elevated MI in Phase~2 is a robust property of the rule table, not a
stochastic artifact of a single initial configuration. The same-state
adjacency fraction (a categorical spatial statistic appropriate for nominal
state data) is also recorded per seed in the multi-seed dataset and confirms
local coordination patterns.

%% =========================================================================
%% C. PSEUDOCODE
%% =========================================================================
\section{Algorithmic Pseudocode}
\label{app:pseudocode}

\paragraph{Simulation loop} (\texttt{world.py:step()}).
At each of the 200 time steps:
\begin{enumerate}
  \item Generate a random permutation of agent indices.
  \item For each agent in order:
    \begin{enumerate}
      \item Observe local neighborhood (von~Neumann, 4 cells).
      \item Compute observation vector $(s, n, d)$ or $(s, n)$ depending on
            phase, where $s$ = own state, $n$ = occupied neighbor count,
            $d$ = dominant neighbor state.
      \item Look up action in shared rule table: $a = T[\mathrm{index}(s, n, d)]$.
      \item Execute action: move (if target cell empty), change state, or no-op.
    \end{enumerate}
\end{enumerate}

\paragraph{Filter checks} (\texttt{filters.py}).
After each step, check:
(1) \emph{Halt}: positions and states unchanged for 10 consecutive steps $\to$ terminate.
(2) \emph{State uniform}: all agents share the same state $\to$ terminate.

\paragraph{MI computation} (\texttt{metrics.py}).
For the final snapshot:
\begin{enumerate}
  \item Enumerate all occupied neighbor pairs (right/down on torus, deduped).
  \item Compute joint and marginal state distributions from pairs.
  \item $\hat{I} = \sum p(s_i, s_j) \log_2 \frac{p(s_i, s_j)}{p(s_i)\,p(s_j)}$.
  \item Apply Miller-Madow correction:
        $\hat{I}_{\mathrm{MM}} = \hat{I} - \frac{K_{\mathrm{joint}} - K_X - K_Y + 1}{2n\ln 2}$.
  \item Clamp to $\geq 0$.
\end{enumerate}

\paragraph{Shuffle null} (\texttt{metrics.py}).
Repeat 200 times: permute states among fixed occupied positions, compute
$\hat{I}_{\mathrm{MM}}$, average.
$\mathrm{MI}_{\mathrm{excess}} = \max(\hat{I}_{\mathrm{MM}} - \overline{I}_{\mathrm{shuffle}}, 0)$.

%% =========================================================================
%% BIBLIOGRAPHY
%% =========================================================================
\footnotesize
\bibliographystyle{apalike}
\bibliography{references}

\end{document}
