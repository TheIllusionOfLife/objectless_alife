\documentclass[letterpaper]{article}
\usepackage{natbib,alifeconf}  %% The order is important
\usepackage{url,hyperref}
\usepackage{amsmath,amssymb}
\usepackage{booktabs}
\usepackage{xcolor}
\usepackage{subcaption}
\usepackage{multirow}

\title{Supplementary Material:\\
Emergent Spatial Coordination from Negative Selection Alone}

\author{Anonymous}

\begin{document}

\maketitle

%% =========================================================================
%% A. DENSITY SWEEP ROBUSTNESS
%% =========================================================================
\section{Density Sweep Robustness}
\label{app:density_sweep}

The main experiments use a fixed density of 7.5\% (30 agents on a
$20 \times 20$ grid). To assess robustness across density levels, we
evaluated both Phase~1 and Phase~2 across 12 density conditions: 3 grid
sizes ($15 \times 15$, $20 \times 20$, $30 \times 30$) $\times$ 4 agent
counts (15, 30, 60, 90), yielding densities from 0.017 to 0.400.
Each condition was evaluated with 600 rules (100 rules $\times$ 6 seed
batches), totaling 14{,}400 rule evaluations, all using the Miller-Madow
bias-corrected MI estimator.

\begin{table*}[htbp]
\centering
\caption{Density sweep results across 12 conditions.
$\mathrm{MI}_{\mathrm{excess}}$ values are
Miller-Madow corrected, bits.  Phase~2 achieves nonzero median
$\mathrm{MI}_{\mathrm{excess}}$ in 8 of 12 conditions, including all
conditions with $\geq 60$ agents, while Phase~1 remains at zero across
all densities tested.}
\label{tab:density_sweep}
\begin{tabular}{rllcccc}
\toprule
Density & Grid & Agents & \multicolumn{2}{c}{Median MI\textsubscript{excess} (bits)} & \multicolumn{2}{c}{Survival (\%)} \\
\cmidrule(lr){4-5} \cmidrule(lr){6-7}
 & & & P1 & P2 & P1 & P2 \\
\midrule
0.017 & $30 \times 30$ & 15 & 0.000 & 0.000 & 86.7 & 86.7 \\
0.033 & $30 \times 30$ & 30 & 0.000 & 0.000 & 83.3 & 85.8 \\
0.037 & $20 \times 20$ & 15 & 0.000 & 0.000 & 80.8 & 81.8 \\
0.067 & $15 \times 15$ & 15 & 0.000 & 0.000 & 68.7 & 76.0 \\
0.067 & $30 \times 30$ & 60 & 0.000 & 0.199 & 70.5 & 76.3 \\
0.075 & $20 \times 20$ & 30 & 0.000 & 0.114 & 70.2 & 72.5 \\
0.100 & $30 \times 30$ & 90 & 0.000 & 0.246 & 67.0 & 74.2 \\
0.133 & $15 \times 15$ & 30 & 0.000 & 0.130 & 62.2 & 64.8 \\
0.150 & $20 \times 20$ & 60 & 0.000 & 0.191 & 63.0 & 69.7 \\
0.225 & $20 \times 20$ & 90 & 0.002 & 0.176 & 64.0 & 76.7 \\
0.267 & $15 \times 15$ & 60 & 0.000 & 0.148 & 63.5 & 72.7 \\
0.400 & $15 \times 15$ & 90 & 0.000 & 0.107 & 68.0 & 83.7 \\
\bottomrule
\end{tabular}
\end{table*}

Table~\ref{tab:density_sweep} and Figures~\ref{fig:mi_density}
and~\ref{fig:surv_density} present the results. Phase~2 achieves nonzero
median $\mathrm{MI}_{\mathrm{excess}}$ in 8 of 12 conditions, including all
conditions with $\geq 60$ agents, peaking near $d = 0.100$
($\mathrm{MI}_{\mathrm{excess}} = 0.246$ bits at $30 \times 30$, 90 agents)
before declining at higher densities. Phase~1 remains at zero median
$\mathrm{MI}_{\mathrm{excess}}$ across all 12 conditions, confirming that the
Phase~2 advantage is not an artifact of the specific grid configuration used
in the main experiments.

Phase~2 also consistently achieves higher survival rates than Phase~1, with
the gap widening at higher densities (e.g., 76.7\% vs.\ 64.0\% at $d =
0.225$; 83.7\% vs.\ 68.0\% at $d = 0.400$).

\begin{figure}[htbp]
\centering
\includegraphics[width=0.85\linewidth]{figures/figA1_mi_vs_density.pdf}
\caption{Median neighbor MI vs.\ agent density for Phase~1 and Phase~2.
Phase~2 MI peaks at medium densities and declines at higher densities, while
Phase~1 remains at zero throughout.}
\label{fig:mi_density}
\end{figure}

\begin{figure}[htbp]
\centering
\includegraphics[width=0.85\linewidth]{figures/figA2_survival_vs_density.pdf}
\caption{Survival rate vs.\ agent density. Phase~2 consistently achieves
higher survival than Phase~1, with the gap widening at higher densities.}
\label{fig:surv_density}
\end{figure}

%% =========================================================================
%% B. MULTI-SEED ROBUSTNESS
%% =========================================================================
\section{Multi-Seed Robustness}
\label{app:multi_seed}

The main experiments evaluate each rule table with a single simulation seed.
To assess whether high-$\mathrm{MI}_{\mathrm{excess}}$ rules are robust
properties of the rule table rather than seed-specific accidents, we selected
the top~50 Phase~2 rules by $\mathrm{MI}_{\mathrm{excess}}$ from the main
experiment and re-evaluated each across 20 independent simulation seeds.

\begin{table}[htbp]
\centering
\caption{Multi-seed robustness of top-50 Phase~2 rules (20 seeds each).}
\label{tab:multi_seed}
\begin{tabular}{lc}
\toprule
Metric & Value \\
\midrule
Rules with median $\mathrm{MI}_{\mathrm{excess}} > 0$ & 41/50 (82.0\%) \\
Mean $P(\mathrm{MI}_{\mathrm{excess}} > 0)$ across seeds & 0.733 \\
Overall survival rate & 76.2\% \\
\bottomrule
\end{tabular}
\end{table}

Table~\ref{tab:multi_seed} shows that 82\% of top Phase~2 rules maintain
positive median $\mathrm{MI}_{\mathrm{excess}}$ across seeds, and on average
73.3\% of seeds per rule produce positive excess MI. This confirms that
elevated MI in Phase~2 is a robust property of the rule table, not a
stochastic artifact of a single initial configuration. The same-state
adjacency fraction (a categorical spatial statistic appropriate for nominal
state data) is also recorded per seed in the multi-seed dataset and confirms
local coordination patterns.

%% =========================================================================
%% C. MORAN'S I (RELOCATED FROM TABLE 1)
%% =========================================================================
\section{Moran's $I$ by Condition}
\label{app:morans_i}

Table~1 in the main text reports the same-state adjacency fraction as the
primary categorical spatial statistic.  For completeness, we report
Moran's~$I$ here, noting that it treats categorical states as numeric
(computing deviations from an arithmetic mean) and is therefore
inappropriate as a primary indicator for nominal data.

\begin{table}[htbp]
\centering
\caption{Median Moran's $I$ by condition (final-step snapshot,
5{,}000 rules per condition).  Moran's $I$ treats states as numeric
and is a secondary indicator; see Table~1 for the categorical
adjacency fraction.}
\label{tab:morans_i}
\begin{tabular}{lc}
\toprule
Condition & Median Moran's $I$ \\
\midrule
Random Walk & $-0.030$ \\
Control     & 0.124 \\
Phase 1     & $-0.011$ \\
Phase 2     & $-0.020$ \\
\bottomrule
\end{tabular}
\end{table}

%% =========================================================================
%% D. HALT WINDOW SENSITIVITY
%% =========================================================================
\section{Halt Window Sensitivity}
\label{app:halt_window}

The main experiments use a 10-step halt window.  To assess sensitivity,
we evaluated the top-50 Phase~2 rules across halt windows of
\{5, 10, 20\} steps.

\begin{table}[htbp]
\centering
\caption{Halt window sensitivity for top-50 Phase~2 rules.
Results are qualitatively unchanged across the tested range.}
\label{tab:halt_window}
\begin{tabular}{ccc}
\toprule
Halt Window & Survival Rate & Median MI\textsubscript{excess} \\
\midrule
5  & 78.0\% & 0.486 \\
10 & 78.0\% & 0.486 \\
20 & 78.0\% & 0.486 \\
\bottomrule
\end{tabular}
\end{table}

Table~\ref{tab:halt_window} confirms that the halt-window parameter has
no impact on the qualitative findings for these top-performing rules:
survival rates (78.0\%) and median MI\textsubscript{excess} (0.486~bits)
are identical across all three tested windows.  This indicates that the
top-50 rules either survive to completion or halt well within the first
5~steps, with no rules in the intermediate regime.

%% =========================================================================
%% E. SURVIVAL RATES WITH CIs
%% =========================================================================
\section{Survival Rates with Confidence Intervals}
\label{app:survival_ci}

\begin{table}[htbp]
\centering
\caption{Survival rates with Wilson score 95\% confidence intervals
(5{,}000 rules per condition).}
\label{tab:survival_ci}
\begin{tabular}{lccc}
\toprule
Condition & Survived / Total & Rate & 95\% CI \\
\midrule
Random Walk & 5000/5000 & 100.0\% & [99.9, 100.0]\% \\
Control     & 2225/5000 & 44.5\%  & [43.1, 45.9]\% \\
Phase 1     & 3570/5000 & 71.4\%  & [70.1, 72.6]\% \\
Phase 2     & 3735/5000 & 74.7\%  & [73.5, 75.9]\% \\
\bottomrule
\end{tabular}
\end{table}

%% =========================================================================
%% F. RANDOM WALK DENSITY SWEEP
%% =========================================================================
\section{Random Walk MI\textsubscript{excess} Across Densities}
\label{app:rw_density}

To confirm that the random walk's $\mathrm{MI}_{\mathrm{excess}}$ remains
near zero regardless of agent density, we extended the density sweep to
include the Random Walk condition.  Across all 12 density conditions
(density range 0.017--0.400), the random walk produces
$\mathrm{MI}_{\mathrm{excess}} \approx 0$ (median $\leq 0.06$~bits),
confirming that its elevated raw MI is entirely attributable to
pair-count bias at all tested densities.

%% =========================================================================
%% G. ALTERNATIVE NULL MODELS
%% =========================================================================
\section{Alternative Null Models}
\label{app:alt_nulls}

In addition to the state-shuffle null used throughout the main text, we
evaluated two alternative null models to assess the robustness of the MI
calibration:

\begin{itemize}
  \item \textbf{Block shuffle}: States are shuffled within spatial blocks
        ($4 \times 4$), preserving local autocorrelation structure while
        destroying inter-block correlations.
  \item \textbf{Fixed-marginal}: Synthetic snapshots are generated with
        identical marginal state distributions but independent spatial
        placement (each position draws independently from the observed
        state frequencies).
\end{itemize}

\begin{table}[htbp]
\centering
\caption{Alternative null model comparison for top-50 Phase~2 rules
(mean MI across 200 null samples per rule).  All three null models
produce substantially lower MI than the observed values, confirming
that Phase~2's elevated MI reflects genuine spatial coordination.}
\label{tab:alt_nulls}
\begin{tabular}{lc}
\toprule
Null Model & Mean Null MI (bits) \\
\midrule
State shuffle (main text) & 0.264 \\
Block shuffle ($4 \times 4$) & 0.899 \\
Fixed-marginal & 0.250 \\
\bottomrule
\end{tabular}
\end{table}

The block-shuffle null produces substantially higher MI (0.899~bits) than
the state-shuffle null (0.264~bits), as expected since it preserves
within-block correlations.  The fixed-marginal null (0.250~bits) is
comparable to the state-shuffle.  In all cases, mean observed MI for the
top-50 Phase~2 rules (1.646~bits) substantially exceeds the null values,
confirming genuine spatial coordination.

%% =========================================================================
%% H. SPATIAL SCRAMBLING
%% =========================================================================
\section{Spatial Scrambling Control}
\label{app:spatial_scrambling}

To confirm that Phase~2's elevated MI depends on agents' specific
positions rather than their state distribution alone, we performed
spatial scrambling: for each rule's final snapshot, we randomly
reassigned occupied positions among agents while keeping their states
fixed ($N = 200$ scrambles per rule).

For top-50 Phase~2 rules, the mean observed MI is 1.646~bits while the
mean scrambled MI drops to 0.270~bits---comparable to the shuffle null
baseline (0.264~bits).  This confirms that the observed MI arises from
genuine local spatial coordination (agents with correlated states being
\emph{near} each other) rather than from the state distribution itself.

%% =========================================================================
%% I. TRANSFER ENTROPY
%% =========================================================================
\section{Transfer Entropy}
\label{app:transfer_entropy}

Mutual information measures symmetric statistical dependence between
neighboring states.  To assess \emph{directional} information flow, we
computed transfer entropy (TE) from neighbor states to agent next-states:
\begin{equation}
  \mathrm{TE} = I(S_j^t ; S_i^{t+1} \mid S_i^t)
\end{equation}
where $S_i^t$ is agent $i$'s state at time $t$ and $S_j^t$ is a
neighboring agent's state.  This measures how much knowing a neighbor's
current state reduces uncertainty about the focal agent's next state,
beyond what the agent's own current state provides.

Miller-Madow bias correction is applied.  For the top-50 rules in each
condition, median TE values are: Phase~2 = 0.072~bits, Phase~1 =
0.003~bits, Control = 0.113~bits.  Phase~2 shows substantially elevated
TE compared to Phase~1, confirming directional information flow from
neighbors to agents.  The Control condition's higher TE reflects its
inclusion of a step-clock dimension that creates temporal state
dependence without genuine spatial coordination (recall that Control
$\mathrm{MI}_{\mathrm{excess}} \approx 0$).

%% =========================================================================
%% J. CAPACITY-MATCHED CONTROLS
%% =========================================================================
\section{Capacity-Matched Controls}
\label{app:capacity_matched}

To further isolate the role of observation \emph{content} from table
\emph{capacity}, we evaluated two additional control conditions:

\begin{itemize}
  \item \textbf{Capacity-matched Phase~1}: 100-entry tables where
        indices are aliased so that all dominant-state values for the
        same (self-state, neighbor-count) pair map to the same action.
        This provides Phase-2-sized tables with only Phase-1-level
        observation content.
  \item \textbf{Random-encoding Phase~2}: 100-entry tables with the
        same alphabet size as Phase~2, but the mapping from neighborhood
        configuration to observation index is randomly permuted.  This
        tests whether the \emph{structure} of the encoding matters
        beyond alphabet size.
\end{itemize}

The capacity-matched Phase~1 control produces median
$\mathrm{MI}_{\mathrm{excess}} = 0.000$ (survival 71.4\%), matching
standard Phase~1 ($\mathrm{MI}_{\mathrm{excess}} = 0.000$) and well below
standard Phase~2 ($\mathrm{MI}_{\mathrm{excess}} = 0.096$).  This
confirms that table capacity alone does not explain the Phase~2
advantage---Phase-1-level observations remain ineffective even with
100-entry tables.  The random-encoding Phase~2 control produces median
$\mathrm{MI}_{\mathrm{excess}} = 0.106$ (survival 75.1\%), comparable to
standard Phase~2, which is expected because the observation encoding
(self-state, neighbor count, dominant state) is identical; only the
table-entry ordering differs, which is irrelevant for randomly generated
tables.

%% =========================================================================
%% K. SPACE-TIME PLOTS
%% =========================================================================
\section{Space-Time Plots}
\label{app:spacetime}

Space-time plots (kymographs) for representative rules from each
condition are available in the code repository.  These show the temporal
evolution of agent states on the grid, providing qualitative visual
evidence for the distinct dynamical regimes described in the main text:
Phase~1's frozen crystallization, Phase~2's sustained dynamic
coordination, and Control's chaotic fluctuations.

%% =========================================================================
%% L. PSEUDOCODE
%% =========================================================================
\section{Algorithmic Pseudocode}
\label{app:pseudocode}

\paragraph{Simulation loop} (\texttt{world.py:step()}).
At each of the 200 time steps:
\begin{enumerate}
  \item Generate a random permutation of agent indices.
  \item For each agent in order:
    \begin{enumerate}
      \item Observe local neighborhood (von~Neumann, 4 cells).
      \item Compute observation vector $(s, n, d)$ or $(s, n)$ depending on
            phase, where $s$ = own state, $n$ = occupied neighbor count,
            $d$ = dominant neighbor state.
      \item Look up action in shared rule table: $a = T[\mathrm{index}(s, n, d)]$.
      \item Execute action: move (if target cell empty), change state, or no-op.
    \end{enumerate}
\end{enumerate}

\paragraph{Filter checks} (\texttt{filters.py}).
After each step, check:
(1) \emph{Halt}: positions and states unchanged for 10 consecutive steps $\to$ terminate.
(2) \emph{State uniform}: all agents share the same state $\to$ terminate.

\paragraph{MI computation} (\texttt{metrics.py}).
For the final snapshot:
\begin{enumerate}
  \item Enumerate all occupied neighbor pairs (right/down on torus, deduped).
  \item Compute joint and marginal state distributions from pairs.
  \item $\hat{I} = \sum p(s_i, s_j) \log_2 \frac{p(s_i, s_j)}{p(s_i)\,p(s_j)}$.
  \item Apply Miller-Madow correction:
        $\hat{I}_{\mathrm{MM}} = \hat{I} - \frac{K_{\mathrm{joint}} - K_X - K_Y + 1}{2n\ln 2}$.
  \item Clamp to $\geq 0$.
\end{enumerate}

\paragraph{Shuffle null} (\texttt{metrics.py}).
Repeat 200 times: permute states among fixed occupied positions, compute
$\hat{I}_{\mathrm{MM}}$, average.
$\mathrm{MI}_{\mathrm{excess}} = \max(\hat{I}_{\mathrm{MM}} - \overline{I}_{\mathrm{shuffle}}, 0)$.

%% =========================================================================
%% BIBLIOGRAPHY
%% =========================================================================
\footnotesize
\bibliographystyle{apalike}
\bibliography{references}

\end{document}
